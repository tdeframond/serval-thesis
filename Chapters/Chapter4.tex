\chapter{FAC  Method and Material}



\section{Test methodology...}
My work methodology is following an Agile method. Every tests can be considered as a single task or a user-story such as my supervisor can be considered as the scrum-master. Each time I chose a new ticket, I try to chose a deadline to fix the ticket according to its complexity. Obviously it is not often easy for me to respect this delay since I am not familiar with the technologies I have to use here. For example, a really simple task can take me a huge amount of time to execute because of my lack of knowledge in these new languages. \\   
The test methodology is not completely defined and can be different depending on the test himself. Nonetheless, the global aspect follows the same basic pattern every time.

\subsection{Test technology}
\subsubsection{Bash}
As Serval project use lots of different languages and devices, we can not use a normal test frameworks like JUnit for example. Indeed we are not doing a test driven development or just testing codes. My goal in this project is to test the whole thing, not just some codes. This i why I need to use something more general and less specialized than a test framework. To this end, I will use basics shell scripts which are very useful to communicate with the Mesh Extender since it is running on a linux kernel. Bash is the most famous and simple shell language that I can use. It allows me to call directly lots of linux command and high level instructions. This is very practical with what concerns Wi-Fi connections for example 


\subsection{Test overlook}
The first task to do before each test is to sort it and classify it between hardware and software categories. This also helps to chose which language will be more adequate to test it and in which environment I have to code it. Most of the time, I write the codes either with a Shell script either with an Expect Script, either with both.  

\section{Simulation methodology...}

Phasellus nisi quam, volutpat non ullamcorper eget, congue fringilla leo. Cras et erat et nibh placerat commodo id ornare est. Nulla facilisi. Aenean pulvinar scelerisque eros eget interdum. Nunc pulvinar magna ut felis varius in hendrerit dolor accumsan. Nunc pellentesque magna quis magna bibendum non laoreet erat tincidunt. Nulla facilisi.

# Thesis outline:

1. Introduction
1.1 Motivation
1.2 Scope
1.3 Research Questions
1.4 Structure of Thesis
2. Literature Review
2.1 Serval Mesh & Serval Mesh Extender
2.1.1 First Generation Serval Mesh Extender
2.1.2 Second Generation Serval Mesh Extender Design
2.2 Testing Methodologies
2.3 Pacific Humanitarian Challenge
[ 3. Method & Materials
3.1 Test methodology ...
3.2 Simulation methodology ... ]
4. Functional Hardware testing of Serval Mesh Extender
4.1 Environmental testing
4.2 Post-assembly testing
5. Functional Software testing of Serval Mesh Extender
6. Simulated Field Operation of Serval Mesh Extender
7. Discussion
8. Conclusions & Future Direction