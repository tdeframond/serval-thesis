\chapter{Functionnal Hardware testing of Serval Mesh Extender}

\section{Test 1 : Boot-loader and flash firmware update}

This is the very first task I have to implement. The goal is to prepare the manufactured Mesh Extender units for the tests and use. In order to do it, we have to install a special Linux distribution on it called OpenWrt. OpenWrt is an embedded operating system based on Linux, primarily used on embedded devices to route network traffic. All components have been optimized for size, to be small enough for fitting into the limited storage and memory available in home routers as the Serval Mesh Extender. \\
In order to install the latest version of the OS, we will use here the command-line interface Shell. This will allow to boot directly the Domino from a laptop. But at this point, we need to make these two entities capable of communicate to each other. That is why we need cables and especially one that can deals with the extender serial port. For this one, we have for the moment a home-made cable that use a D-SUB 25 pins male port on one side and whatever we want on the other side by cutting the edge of the cable and weld components on the desired nude wires. That is why we have welded here a serial port to USB adapter and a power plug to power the PCB. \\
So now we can talk to the PCB with the laptop through command lines. But there is a problem remaining. Serial port is too slow to transfer data and at some point, we will have to download on the Domino the OpenWrt software binaries. That is why we also use an ethernet cable to install the distribution in addition to the serial cable.
\\ 

\subsection{Procedure}
The procedure is simple, we first connect the laptop and the Mesh Extender with the serial and ethernet cables. Then we plug the power cable in order to make the PCB run. We have to install the Serval software which create the OpenWrt files to transfer to the PCB on a linux environment. To this end, we have to install lots of dependencies like for example GNU awk, SVN, OpenSSL library ...etc. That is why we need to install VirtualBox on Macintosh laptop first. Then, just clone the repository OpenWrt from the Serval Project Github source and run the following commands :\\
> ./scripts/feeds update serval\\
> ./scripts/feeds install -p serval\\
> ./scriptes/feeds install -a serval\\
> make world (long time running...)\\ \\
Therefore, the binary for installation should then be in: bin/ar71xx/openwrt-ar71xx-generic-gl-ar150-initramfs-kernel.bin \\
To flash the PCB, we have to connect to the serial port at 115200 by typing:\\

> cu -l /dev/cu.usbserial -s 115200\\

or if we have trouble with cu: \\ 

> screen /dev/cu.usbserial 115200 \\

In either case, we will have to reboot the node in some way, so that we see the uboot prompt. Then we have to press any key to interrupt the boot process.\\
The Mesh Extender node will have an IP of 192.168.1.1, so we should pick another IP address on that subnet for the connected computer and type "httpd" to start the firmware update webserver. Then, just browse to http://192.168.1.1 and select the firmware file to upload, and trigger the firmware update. \\

\subsection{Code}
Here is the expect script for that : \\
\begin{lstlisting}
#!/usr/bin/expect
#Expect script installing an openWrt image on the Serval Mesh Extender

set timeout -1
log_user 0

spawn cu -l /dev/cu.usbserial -s 115200

expect {
	"Connected." 
	{
	 	puts "\n##############################################\n# MESH EXTENDER FIRMWARE UPDATER\n#\n# 1/7 > Please, boot or reboot the PCB" 
	}
}

expect {
	"Hit any key" 
	{ 
		send "\r" 
		puts "# 2/7 > Autoboot well interrupted" 
	}
}

expect {
	"uboot>" 
	{ 
		send "httpd\r" 
		puts "# 3/7 > Server HTTP well started" 
	}
}

expect {
	"HTTP server is ready!"
	{ 
		puts "# 4/7 > Uploading the image..." 
		system curl --silent -o /dev/null -F 'firmware=@./openwrt-ar71xx-generic-gl-ar150-squashfs-sysupgrade.bin' -F 'filename=\$openwrt-ar71xx-generic-gl-ar150-squashfs-sysupgrade.bin' http://192.168.1.1/ 
	}
}

expect {
	"upload is done!"
	{ 
		puts "# 5/7 > Upload successful\n# 6/7 > Upgrading the firmware... DO NOT POWER OFF " 
	}
}

expect {
	"done!"
	{ 
		puts "# 7/7 > HTTP upgrade is done! Rebooting..." 
	}
}

expect {
	"Hit any key" 
	{ 
		send "\r" 
		puts "#\n# FIRMWARE SUCCESSFULLY UPDATED, ENJOY!\n##############################################\n" 
	}
}
\end{lstlisting} 

\subsection{Output and results}
\begin{lstlisting}
##############################################
# MESH EXTENDER FIRMWARE UPDATER
#
# 1/9 > Please, boot or reboot the PCB
# 2/9 > Autoboot well interrupted
# 3/9 > Server HTTP well started
# 4/9 > Uploading the image...
# 5/9 > Upload successful
# 6/9 > Upgrading the firmware... DO NOT POWER OFF 
# 7/9 > HTTP upgrade is done! Rebooting...
# 8/9 > Please reboot the node one last time in order to finish the installing process
# 9/9 > Second and last reboot. Almost done... 
#
# FIRMWARE SUCCESSFULLY UPDATED, ENJOY!
##############################################
\end{lstlisting} 

We now have a brand new and up to date Serval Mesh Extender. If something goes wrong in the installation, the script just exit and tell you to launch the test again. 

\section{Test 2 : Wireless connections}

The second automatic test i have to implement are related to the Mesh networks. Indeed we have to make sure that, after an upgrade or any random boot, the Mesh Extender will provide the expected networks. That means two wireless connections and one Ethernet connection. The first wireless connection is actually a hotspot on which every device can connect. The second one is the ad-hoc peer connection in order to communicate with other Mesh Extenders. This is the one who will diffuse all the Rhizome messages. Then the Ethernet connection is here to enable the transfer of data on the Mesh from a laptop. It will be a way to connect whatever else devices locally to the Extender as well. \\

\subsection{Procedure}
This test is a bit more complicated than the first one because it needs the output of the program to react according to it. This is a highest level of expect and did a lot of research to finally understand how we can fix this issue. Actually with expect, we can definitely not read the output of a shell. Or at least not directly. What we can though is to record or redirect this output into a file and then, read this file with expect. It is not very practical, it consumes time and memory but this is the only way... \\
In this program, first of all i need to check the presence of the wireless networks. For this purpose, i use the airport tool available in the Unix environment. There in one little problem though. The airport software is most of the time already installed on laptops but we have to add it to the global PATH before we can use it. Once this is done, we can check for the Wifi networks from the shell with the command : \\
> airport -s \\
We filter the result of it by piping the output with the grep command which i redirect directly into a file in order to use it with expect after : \\
> airport -s | grep servalproject > networks-found  \\
Now, we have to check if the networks-found file contains indeed the networks we are looking for. We also have to make sure that, if there is many Mesh Extender working at the same time, the networks we are looking at is really the one from the Mesh Extender we are testing and not the one from another. For this, we can just compare the mac address of both the Mesh Extender and the one from the Wifi network. For testing the test script, i just simulate and assume that the mac-address of the testing Mesh Extender will be in an other file. I just have then to open both of these files and compare them as simple strings in iterative loops. If both of the addresses match, test is successful. For this test I have thus to make use of file writing and access procedures in the TCL language which is really new for me. 

\subsection{Code}
\begin{lstlisting}
#!/usr/bin/expect
#Expect script checking the wifi and ethernet connections of the Serval Mesh Extender

set timeout -1
#log_user 0

spawn cu -l /dev/cu.usbserial -s 115200

expect {
	"Connected." 
	{
	 	puts "\n##############################################\n# MESH EXTENDER WIRELESS CONNECTIONS TEST \n#\n# 1 > Please, boot or reboot the PCB" 
	}
}

puts "\nCHECKPOINT1\n"

expect {
	"Starting kernel..."
	{
	 	puts "# Please wait while the Mesh Extender is booting..." 
	}
}

puts "\nCHECKPOINT2\n"

expect {
	"nonblocking" 
	{
	 	puts "# Mesh Extender started. Searching for Serval networks..."
	 	system airport -s | grep servalproject > networks-found
	}
}

set results(wifi-AP) "fail"
set results(wifi-AH) "fail"
set results(ethernet) "fail"

#read the file
set file [open "networks-found"]
set content [read $file] 
close $file
set file2 [open "mac-address"]
set macs [read $file2] 
close $file2

#exctract lines
set lines [split $content "\n"]

##Iterate over the lines
foreach line $lines {

    ## Split into fields on colons
    set fields [split $line " "]
	set stripped [lsearch -inline -all -not -exact $fields {}] 
    lassign $stripped \
            networkName macAddress

    puts "Network name = $networkName and mac address is $macAddress"

    #tests
    if { $networkName == "mesh.servalproject.org"} {
    	set results(wifi-AH) "succes"
    }
    if { $networkName == "public.servalproject.org" } {
    	set results(wifi-AP) "succes"
    }
}

parray results
\end{lstlisting}

\subsection{Output and results}
\begin{lstlisting}
##############################################
# MESH EXTENDER WIRELESS CONNECTIONS TEST
#
# 1/5 > Please, boot or reboot the PCB
# 2/5 > Please wait while the Mesh Extender is booting...
# 3/5 > Mesh Extender started, looking for Serval networks...
# 4/5 > Wifi-AP "succes"
# 5/5 > Wifi-AH "succes"
#
# END OF THE TEST
##############################################
\end{lstlisting}

Wifi-AP stands for the wifi access point which has got in this particular test this MAC address : public.servalproject.org e6:95:6e:40:5b:e2 .
\\
Wifi-AH stands for the wifi Ad-Hoc network which has got in this particular test this MAC address : mesh.servalproject.org 02:ca:ff:dd:ca:ce .
\\
If the script does find the required network, it goes to "success" status. If not, it goes obviously to "failed" status. 

\section{Tests in development}

\subsection{Test 3 : Ethernet connection}
This test is making sure that the Ethernet connection is working and operating properly. To this end I connect the PCB to the laptop again but with an Ethernet cable this time. The script will try to ping and open a SSH session between the laptop and the Mesh Extender. Once connected, it will try to run a couple of commands to make sure the SSH tunnel is usable. 
\par
Actually I did not finish to implement this test because I could not manage to ping the PCB nor succeed to open an SSH session. There must be a kind of protection or incompatibility with the devices. My supervisor told me that the next version of the OpenWRT image for Serval will probably fix it by allowing SSH connections. For now I am stuck and I just know that the Ethernet connection works well but without knowing how to test it. I give this next steps to my French fellows for the next semester. 

\subsection{Test 4 : Serval server}

Now that we know that the Serval access point is running, we have to test the server it is actually running. For this, I will just run a Bash script who will call some curl commands in order to fetch the main urls from the server and test them.
\\
The server is actually reachable on the IP : 10.0.0.1 and depending on the port we try to access, we have different pages. Likewise we can also access different application on this server depending on the speed connection of the Ethernet connection we established with the 'cu' Linux command. If I want to test the normal aspect of the Mesh Extender and launch Uboot, we need the speed of 115200. If now I want to access the Lbard application I have to double the speed and get to 230400.
\\
On this server, we can also have informations about the hardware and other specificities of the Mesh Extender like information about the SD card mounting or the state of some others services. This is actually quiet useful for the tests but hard to exploit as this is presented as HTML documents which are not easy to deal with in a script. 
