\chapter{Literature Review}

\section{The Serval Project}

\subsection{What is Serval ?}
Serval is a humanitarian open source project aiming at providing means of communication to people in the incapacity to of having cellular coverage. "Cellular communication is great when it's available" but it is not always the case. This is why Serval offers an independent and autonomous cellular network which works with mesh extenders and normal smartphones without the need of external cellular infrastructures. \\
This project was created in 2010 by my supervisor Paul Gardner-Stephen and Romana Challans in response to the terrible and disastrous earthquake that happened in Haiti just before. Years after years, many people join the team to work and help on the project which is based in the telecommunications laboratory of Flinders University in Tonsley, Mitchell Park, SA. 


\subsection{Why Serval ?}
Serval is here to help with one main purpose: COMMUNICATE, regardless the conditions and circumstances.
\begin{itemize}
\item Communicate anytime, even when the usual and main phone network is down or the cellular infrastructures are broken.
\item Communicate anywhere, as cellular networks are not available everywhere. For instance, 75\% of Australia is not covered as lots of poor countries in the world. 
\item Communicate privately, with end-to-end encryption for phones calls and SMS.
\end{itemize}

\subsection{Serval purposes}
As cellular network is not everywhere and always accessible, the Serval Project is about creating an alternative communication network that does not depend on a big, official and expensive infrastructure. That means they do not need a license neither nor a carrier. In this conditions, Serval is using a Mobile Ad-Hoc Network (MANETs) which is appropriate for this. \\
An other goal of the Project is to connect sensors to this network in order to monitor wide remote area. For example, a farmer wants to know if a door is open on the other side of his huge farmland. 


\section{Testing Methodologies}

Phasellus nisi quam, volutpat non ullamcorper eget, congue fringilla leo. Cras et erat et nibh placerat commodo id ornare est. Nulla facilisi. Aenean pulvinar scelerisque eros eget interdum. Nunc pulvinar magna ut felis varius in hendrerit dolor accumsan. Nunc pellentesque magna quis magna bibendum non laoreet erat tincidunt. Nulla facilisi.

\section{Pacific Humanitarian Challenge}

Duis eget massa sem, gravida interdum ipsum. Nulla nunc nisl, hendrerit sit amet commodo vel, varius id tellus. Lorem ipsum dolor sit amet, consectetur adipiscing elit. Nunc ac dolor est. Suspendisse ultrices tincidunt metus eget accumsan. Nullam facilisis, justo vitae convallis sollicitudin, eros augue malesuada metus, nec sagittis diam nibh ut sapien. Duis blandit lectus vitae lorem aliquam nec euismod nisi volutpat. Vestibulum ornare dictum tortor, at faucibus justo tempor non. Nulla facilisi. Cras non massa nunc, eget euismod purus. Nunc metus ipsum, euismod a consectetur vel, hendrerit nec nunc.