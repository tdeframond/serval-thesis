\chapter{Introduction}

This chapter of my thesis is the "why?" of my project. I am explaining here what are my motivations that leaded me to this and why i wanted to chose this work as my end study project instead of an other one. The scope and the context of this experiment are also part of this chapter. The two last points i am discussing here concern all my research questions and the global structure of my thesis.

\section{Motivation}

\subsection{Australia}
Of course and since I live in France, an important motivation was to travel and discover Australia. But as this "touristic" motivation is not very appropriate in this thesis I will not develop it that much. But I still want to say that traveling is essential for a human to grow and get mature. This apply for becoming a engineer as well. Now that the semester is almost finished, I can tell that this abroad experiment has helped me to get prepared for my future life by making me think in a different way and more.\\

\subsection{Telecommunication}
Second motivation is the field of telecommunications. This is my studies and what I am interested in. It was thus obvious that this project will fit to me. Indeed, the Serval Project concerns an emergency phone network. It deals thus with antennas, communication protocols and signal processes. All this topics will be helpful for my future jobs. All what I learned in the past years in my school, I can apply it here on this project which is totally on the continuous of my studies track. 

\subsection{Laboratory and Research}
In my five years of study, I have never really worked in a laboratory as a researcher. I have done many tutorials and practical works with my class but it was just study and nothing to see with research. I wanted thus to discover the world of research and how all of this work. How is running a laboratory ? Who is in charge of it ? What are the relationships in it ? How are the projects financed ? I am really happy now to know much more about this part of the engineer world. 

\subsection{Humanitarian}
Last and not the least, I really appreciate the humanitarian aspect of the project. Indeed, as I will explain in the next section, the Serval Project is here to help poor islands from the pacific to recover after a natural disaster for example. It can also be the first step for poor villages to have access to distant communication without having to invest into expensive and permanent infrastructures. If I can work for a good and human purpose, it is better. I am really happy to help in this adventure rather to work for big industries who just care about money despite the environment or human conditions. 

\section{Scope}
The Serval Project is a suite of technologies designed to facilitate and sustain mobile telecommunications in the absence of supporting infrastructure, such as cellular networks or electricity. \par The two main components of the Serval Project are the Serval Mesh Extender Hardware and the Serval Mesh App. Basically the Serval Mesh Extender is a low-cost communications relay device that extends the range of communications among phones using Wifi technology. The laboratory has a partnership with my university in France named INSA de Lyon so that each year, French students can help on the project as a one semester exchange program. I am one of these student. \\ 

For the year 2017, the Australian Department of Foreign Affairs and Trade have commissioned the University to pilot Serval in the Pacific. Consequently, we have to prepare the Serval Mesh Extender technologies for field use in tropical-maritime environments, and without any dependencies on mains electricity. To this end the first Serval Mesh Extender is being redesigned to satisfy these requirements. However, this process is not yet complete. \par

Therefore there is an need to devise and apply a testing regime for the new Serval Mesh Extender design, to ensure that it meets the necessary functional requirements, and that the units are possible to easily manufacture. The focus of this project will be on the creation and application of such test protocols, to ensure that the Serval Mesh Extender devices are ready for deployment in the field pilot.

\section{Research questions}
For this work I am following an Agile methodology. Therefore I have to organize my experiments in an iterative way and plan them with precise steps and precise goals. A list of the automatic tests I have to implement will help me with this. \\
The tests must be automatic and quick. That is why I need to implement them in a logical and low-level software environment. Basically I will connect the Mesh Extenders with a laptop and then run all the tests on it. In the end, it will display all the results so we can know if the extenders are ready to use or if we have to do some changes on the settings and if the manufacturing of the extender is okay. \\
Finally and with the help of my supervisor instructions, I used Expect and Shell scripts to code my test. 
These are the different parts of the Serval Project I will have to implement tests for : 
\begin{itemize}
\item Mesh Extender hardware 
\item Mesh Extender cables
\item Mesh Extender software 
\item Mesh Extender network functions
\item end-to-end connection testing with various topologies
\item manufacturing quality control 
\item acceptance testing
\end{itemize}
Since we have to be ready for the Vanuatu expedition which will happen in May, work must be done in two or three months. That makes thus two or three tests by months. According to the fact I have to see and learn how the Mesh Extender operates and the callback I need in Expect and Shell language. 


\section{Structure of Thesis}
The structure of my thesis will follow a height points pattern. After this introduction, I'm going to do some literature review in order to get the context of the Serval Project and to know exactly where about we are in it. I will deal with the different versions of the Serval Mesh extender and introduce the new design of the last version. Then, the methodology will be an other part follow by the Pacific Humanitarian Project which is the reason why we can run this project at this level now. The next big part is the hardware tests split into two categories : the Environmental Testing and the Post-Assembly testing. The first one concerns the abilities of the Mesh Extender to run into a certain environment, for example rain or dust. The second one is more about the software and utility requirements. Next part is obviously the software tests. I will finally present the simulated field operation of the Mesh Extender before discussing all the project and to conclude introduce the future direction of the Serval Project.